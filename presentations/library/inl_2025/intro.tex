

%=================================================================
% Outline
%=================================================================
%\section{Introduction}
%% Auto-generate the TOC slide(s)
\begin{frame}
  \tableofcontents[currentsection]
  %\tableofcontents
\end{frame}

\section*{Outline}% Make it easy to jump to this page in the PDF

% use outline_currentsection.tex to highlight the current section

% Auto-generate the TOC slide(s)
\begin{frame}
  %\tableofcontents[currentsection]
  \tableofcontents
\end{frame}





\section{Introduction}
%%%%%%%%%%%%%%%%%%%%%%%%%%%%%%%%%%%%%%%%%%%%%%%%%
\frame
{
  \frametitle{History}

  \begin{center}
  \includegraphics[width=.3\textwidth]{cfdlab}
  \end{center}

  \begin{itemize}
    \item CFDLab, University of Texas at Austin
      \begin{itemize}
        \item Computational Fluid Dynamics research, from creeping
          (Roy Stogner) to incompressible (John Peterson) to hypersonic
          (Benjamin Kirk)
        \item Adaptive Finite Element Method research (Dr. Graham F.\ Carey)
        \item High-performance computing research (Bill Barth)
        \item Software engineering experience (Robert McLay)
        \item<3-> Stubbornness (Benjamin Kirk, John Peterson, Roy Stogner, ...)
      \end{itemize}
    \item<2-> ``No one ever got a PhD from here for writing a code.'' - Dr. Graham F.\ Carey
  \end{itemize}  


}



%%%%%%%%%%%%%%%%%%%%%%%%%%%%%%%%%%%%%%%%%%%%%%%%%
\frame
{
  \begin{center}
  \includegraphics[width=.3\textwidth]{cfdlab}
  \end{center}

  \begin{block}{The standard Ph.D.-candidate software development
    process:}
  \pause
  \begin{itemize}[<+->]
    \item Create pseudocode
      \begin{itemize}[<+->]
        \item (throw it away; it won't match the real code you
          end up writing)
      \end{itemize}
    \item Create Matlab prototypes, fast
      \begin{itemize}[<+->]
        \item You're just solving one problem
        \item Don't waste time on
          documentation for slow-witted users
        \item You're just going to rewrite the bad hacks
          anyway
      \end{itemize}
    \item Rewrite from scratch (maybe in Fortran?)
      \begin{itemize}[<+->]
        \item ``the single worst strategic mistake'' - Joel Spolsky, 2000
        \item ``Other slow-witted users'' includes ``myself, 2 years
          later''
        \item Bad hacks may now be load-bearing dependencies
      \end{itemize}
    \item Graduate, throw it all away, go work on a ``real'' code
  \end{itemize}  
  \end{block}


}


 

%%%%%%%%%%%%%%%%%%%%%%%%%%%%%%%%%%%%%%%%%%%%%%%%%
\frame
{
  \begin{center}
  \includegraphics[width=.3\textwidth]{cfdlab}
  \end{center}

  \begin{block}{libMesh Development Ingredients}
  \pause
  \begin{itemize}[<+->]
    \item Inspiration (Wolfgang Bangerth + deal.II, Robert McLay + MGFLO)
    \item Genius (Ben Kirk, John Peterson, 2002)
    \item Collaboration (Daniel Dreyer, Steffen Petersen, 2003; Roy
      Stogner 2004; David Knezevic 2005; Derek Gaston 2006)
    \item Flexibility
      \begin{itemize}[<+->]
        \item Documentation: teaching is the best way to learn
        \item Abstraction: purpose first, algorithm second, data third
          \begin{itemize}[<+->]
            \item C++: OOP \emph{and} ``only pay for what you use''
          \end{itemize}
        \item Encapsulation: only promise what you can guarantee
        \item Modularity: only use what you must demand
        \item Testing: don't break what you promise
      \end{itemize}
  \end{itemize}  
  \end{block}


}


 




%%%%%%%%%%%%%%%%%%%%%%%%%%%%%%%%%%%%%%%%%%%%%%%%%
\frame
{
  \frametitle{The \libMesh{} Software Library}
  \begin{itemize}
    \item In 2002, the \libMesh{} library was created with these ideas
      and concerns in mind.
    \item Primary goal is to provide individual tools: data structures
      and algorithms that can be shared by widely differing physical
      applications, that may need some combination of
      \begin{itemize}
      \item Implicit numerical methods
      \item Adaptive mesh refinement techniques
      \item Parallel computing 
      \end{itemize}
    \item Unifying theme: \emphcolor{mesh-based simulation of partial differential equations (PDEs)}.
      \begin{itemize}
      \item Continuous and discontinuous Finite Element Methods
      \item Finite Volume Methods
      \item Even many ``mesh-free'' methods
      \end{itemize}
  \end{itemize}
}




 

%%%%%%%%%%%%%%%%%%%%%%%%%%%%%%%%%%%%%%%%%%%%%%%%%
\frame
{
  \frametitle{The \libMesh{} Software Library}

  \begin{block}{A Toolkit, not a Framework}
    \begin{itemize}
      \item \libMesh{} was designed to give students, researchers,
        scientists, and engineers tools to \emphcolor{develop
        simulation codes} or \emphcolor{rapidly implement a numerical
        method}.
      \item \libMesh{} is not an application.
      \item It does not ``solve problem XYZ.''
        \begin{itemize}
          \item It was designed to help \cancel{users} researchers
            create their own application to solve problem XYZ,
            quickly, with advanced numerical algorithms on
            high-performance computing platforms.
          \item It has since been used to create professional
            application frameworks to solve many problems, combined,
            with user-friendly interfaces.
        \end{itemize}
    \end{itemize}    
  \end{block}
} 


%%%%%%%%%%%%%%%%%%%%%%%%%%%%%%%%%%%%%%%%%%%%%%%%%
\frame
{
  \frametitle{Software Reuse}
  \begin{itemize}
    \item When \libMesh{} was created in 2002, many high-quality
      software libraries implemented some fraction of the end-to-end PDE simulation process:
      \begin{itemize}
        \item Linear algebra (Laspack, PETSc)
        \item Partitioning algorithms for domain decomposition
        \item Visualization of solution files
        \item \ldots
      \end{itemize}
    \item \libMesh{} tries to provide flexible, extensible, abstract interfaces to existing software when possible.
    \item We implemented ``glue'' to these pieces, plus what was missing at the time:
      \begin{itemize}
        \item \emphcolor{Flexible data structures for discretizating of spatial domains and systems of PDEs posed on them.}
      \end{itemize}          
  \end{itemize}  
}




\begin{frame}{libMesh Community}
\begin{columns}
\column{.4\textwidth}
\begin{block}{Scope}
\begin{itemize}
\item Free, Open source
\begin{itemize}
\item LGPL2 for core
\end{itemize}
\item 153 Ph.D.\ theses from users, 2254 papers (240+ in 2024)
\item 15 contributors in 2024, 102 historically
\end{itemize}
\end{block}

\column{.6\textwidth}
\includegraphics[width=.45\textwidth]{ablating_hs_wbg}
\includegraphics[width=.25\textwidth]{sov}
\includegraphics[width=.25\textwidth]{marmot1b}
\end{columns}

\begin{columns}
\column{.35\textwidth}
\includegraphics[width=\textwidth]{cloc_libmesh}

\column{.65\textwidth}
\begin{block}{Challenges}
\begin{itemize}
\item Radically different application types
\item Widely dispersed core developers
\begin{itemize}
\item At peak: INL, UT-Austin, U.Buffalo, JSC, MIT, Harvard, Argonne
\end{itemize}
\item OSS, commercial, private applications
\end{itemize}
\end{block}
\end{columns}

\end{frame}


\begin{frame}[t]
  \begin{columns}
    \column{.6\textwidth}
    \begin{center}
      \includegraphics[width=.9\textwidth]{mytreeandroots_allnames}
    \end{center}
    \column{.35\textwidth}
    \begin{itemize}
      \item Foundational (typically optional) library access via LibMesh's ``roots''.
      \item Application ``branches'' built off the library ``trunk''.
      \item Additional middleware layers (e.g. Akselos, GRINS, MOOSE) for more complex applications
    \end{itemize}
  \end{columns}

\end{frame}



\section{Library Design}


\begin{frame}
\frametitle{Geometric Element Classes}

\begin{columns}
\column{.55\textwidth}
\begin{center}
\vspace{-5mm}
\includegraphics[width=.75\textwidth]{DofObjects}
\end{center}
\column{.45\textwidth}
\begin{itemize}
\item Abstract interface gives mesh topology
\item Concrete instantiations of mesh geometry
\item Hides element type from most applications
\item Runtime polymorphism allows mixed element types, dimensions
\item Base class data arrays allow more optimization, inlining
\end{itemize}

\end{columns}

\end{frame}


%%%%%%%%%%%%%%%%%%%%%%%%%%%%%%%%%%%%%%%%%%%%%%%%%
\frame
{
  \frametitle{Linear Algebra}
  \begin{center}
    \includegraphics[width=\textwidth,trim=7.56in 0 0 0,clip]{classlibMesh_1_1SparseMatrix__inherit__graph}
  \end{center}
}



%%%%%%%%%%%%%%%%%%%%%%%%%%%%%%%%%%%%%%%%%%%%%%%%%
\frame
{
  \frametitle{I/O formats}
  \begin{center}
    \includegraphics[height=0.9\textheight]{mesh_io}
  \end{center}
}


%%%%%%%%%%%%%%%%%%%%%%%%%%%%%%%%%%%%%%%%%%%%%%%%%
\frame
{
  \frametitle{Domain Partitioning}
  \begin{center}
    \includegraphics[width=.3\textwidth]{part_trans}
    %\\
    \includegraphics[width=.3\textwidth]{streamtraces}
  \end{center}  

  \includegraphics[width=.65\textwidth]{partitioner}
}


%%%%%%%%%%%%%%%%%%%%%%%%%%%%%%%%%%%%%%%%%%%%%%%%%
\begin{frame}
\frametitle{Mesh Data Structures}
\begin{columns}
\column{.6\textwidth}
\begin{center}
\includegraphics[width=.95\textwidth]{MeshUML}
\end{center}
\column{.4\textwidth}
%\begin{block}{}
\begin{itemize}
\item \texttt{MeshBase} gives node or element iterators, all vs active, global vs local
\item \texttt{ReplicatedMesh} or \texttt{DistributedMesh} manages synchronized or distributed data
\end{itemize}

\includegraphics[width=.75\textwidth]{ParallelMesh3}
%\end{block}
\end{columns}

\end{frame}



%%%%%%%%%%%%%%%%%%%%%%%%%%%%%%%%%%%%%%%%%%%%%%%%%
\frame
{
  \frametitle{Discretization: Finite Elements}
  \begin{center}
    \includegraphics[width=0.9\textwidth,trim=7.4in 0 0 0,clip]{classlibMesh_1_1FEAbstract__inherit__graph}
  \end{center}
}      



%%%%%%%%%%%%%%%%%%%%%%%%%%%%%%%%%%%%%%%%%%%%%%%%%
\frame
{
  \frametitle{Algorithms: Error Estimation}
  \begin{center}
    \includegraphics[width=\textwidth]{classlibMesh_1_1ErrorEstimator__inherit__graph}
  \end{center}
}


