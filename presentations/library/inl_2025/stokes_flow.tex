\subsection*{Stokes Flow}
\begin{frame}[t]  
  \begin{block}{}
    \begin{itemize}    
    \item{For multi-variable systems like Stokes flow,
      \begin{equation}
	\begin{array}{rcl}
	  \nonumber
	  %\frac{\partial \bv{u}}{\partial t} +
	  %\left(\bv{u} \cdot \nabla\right) \bv{u} +
	  \nabla p - \nu \Delta \bv{u}  &=& \bv{f}
	  \\
	  \nonumber
	  \nabla \cdot \bv{u} &=& 0
	\end{array}  \hspace{.25in}  \in \hspace{.1in} \Omega \subset \mathbb{R}^2
      \end{equation}
    }
\vspace{-.25in}
      
    \item{The element stiffness matrix concept can extended to include sub-matrices
      \begin{eqnarray}
	\nonumber
	\label{eqn:Ke_stokes}
	\left[
	  \begin{array}{cc|c}
	    \alert<2>{K^e_{u_1 u_1}}   & K^e_{u_1 u_2}             &  K^e_{u_1 p}        \\
	    K^e_{u_2 u_1}              & \alert<3>{K^e_{u_2 u_2}}  &  K^e_{u_2 p} \\ \hline
	    K^e_{p u_1}                & \alert<4>{K^e_{p u_2}}    &  K^e_{p p}      \\
	  \end{array}
	  \right]
	\left[
  \begin{array}{c}
    U^e_{u_1} \\
    U^e_{u_2}\\ \hline
    U^e_{p}
  \end{array}
  \right]-
\left[
  \begin{array}{c}
    \alert<6>{F^e_{u_{1}}} \\
    \alert<7>{F^e_{u_{2}}} \\ \hline
    F^e_{p}
  \end{array}
  \right]
      \end{eqnarray}
    }


      \item
	{
	  \only<1-4>	      {We have an array of submatrices:}
	      \only<1>	      {\texttt{Ke[ ][ ]}}
	      \only<2>	      {\hspace{-0.05in}\texttt{Ke[\alert<2>{0}][\alert<2>{0}]}}
	      \only<3>	      {\hspace{-0.1in}\texttt{Ke[\alert<3>{1}][\alert<3>{1}]}}
	      \only<4>        {\hspace{-0.15in}\texttt{Ke[\alert<4>{2}][\alert<4>{1}]}}

      \only<5->          { 	  And an array of right-hand sides: }
 	\only<5> {\texttt{Fe[]}.}
	\only<6> {\hspace{-0.05in}\texttt{Fe[\alert{0}]}.}
	\only<7> {\hspace{-0.1in}\texttt{Fe[\alert{1}]}.}
	}

	
%%       \only<1>
%% 	  {
%% 	  \item{
%% 	    We have an array of submatrices
%% 	    \texttt{Ke[ ][ ]}.
%% 	  }
%% 	  }

%% 	  \only<2>
%% 	  {
%% 	  \item{
%% 	    We have an array of submatrices
%% 	    \texttt{Ke[\alert<2>{1}][\alert<2>{1}]}.
%% 	  }
%% 	  } 
%%           \only<3>
%%           {
%%  	  \item{
%%  	    In this case, we have an array of submatrices
%%  	    \texttt{Ke[\alert<3>{2}][\alert<3>{2}]}.
%%  	  }
%% 	  }
%%           \only<4>
%%           {
%%  	  \item{
%%  	    In this case, we have an array of submatrices
%%  	    \texttt{Ke[\alert<4>{3}][\alert<4>{2}]}.
%%  	  }
%% 	  }
%%           \only<5>
%%           {
%%  	  \item{
%%  	    And an array of right-hand sides
%%  	    \texttt{Fe[]}.
%%  	  }
%% 	  }
%% 	  \only<6>
%% 	  {
%%  	  \item{
%%  	    And an array of right-hand sides
%%  	    \texttt{Fe[\alert{1}]}.
%%  	  }
%% 	  }
    \end{itemize}
  \end{block}
\end{frame}




\begin{frame}[fragile] 
  \begin{block}{}
    \begin{itemize}    
    \item{The matrix assembly can proceed in essentially the same way.}
    \item{For the momentum equations:}
    \end{itemize}
  \end{block}
  \small
\begin{semiverbatim}
for (q=0; q<Nq; ++q) 
  \alert{for (d=0; d<2; ++d)}
    for (i=0; i<Ns; ++i) \{
      Fe\alert{[d]}(i) += JxW[q]*f(xyz[q],\alert{d})*phi[i][q];
      
      for (j=0; j<Ns; ++j)
        Ke\alert{[d][d]}(i,j) +=
	            JxW[q]*nu*(dphi[j][q]*dphi[i][q]);
    \}
\end{semiverbatim}
\end{frame}
