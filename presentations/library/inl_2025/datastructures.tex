\section{Key Data Structures}

%%%%%%%%%%%%%%%%%%%%%%%%%%%%%%%%%%%%%
\subsection{The Mesh Class}

\begin{frame}
  \frametitle{Operations on Objects in the \texttt{Mesh}}
  \begin{block}{}
    \begin{itemize}
    \item From a \texttt{Mesh} it is trivial to access ranges of objects of interest through \emph{iterators}.
    \item Iterators are simply a mechanism for accessing a range of objects.
    \item \libMesh{} makes extensive use of \emph{predictated iterators} to access, for example,
      \begin{itemize}
        \item All elements in the mesh.
        \item The ``active'' elements in the mesh assigned to the local processor in a parallel simulation.
        \item The nodes in the mesh.
      \end{itemize}
  \end{itemize}
  \end{block}
\end{frame}

\begin{frame}[shrink]
  \frametitle{Mesh Ranges and Iterators}
  \lstinputlisting{snippets/active_elem_iterators.cxx}
\end{frame}

\begin{frame}[shrink]
  \frametitle{Mesh Ranges and Iterators}
  \lstinputlisting{snippets/node_iterators.cxx}
\end{frame}



%%%%%%%%%%%%%%%%%%%%%%%%%%%%%%%%%%%%%
\subsection{The EquationSystems Class}
\begin{frame}
  \frametitle{EquationSystems}
  \begin{block}{}
    \begin{itemize}
      \item A \texttt{Mesh} is a discrete representation of problem geometry.
      \item For each \texttt{Mesh}, there can be an
        \texttt{EquationSystems} object, representing one or more
        coupled systems of equations posed on the \texttt{Mesh}.
        \begin{itemize}
          \item There is at most one \texttt{EquationSystems} object per \texttt{Mesh}.
          \item The \texttt{EquationSystems} object can hold many \texttt{System} objects, each representing a logical system of equations.
        \end{itemize}
      \item High-level operations such as solution input/output
        typically operate at the \texttt{EquationSystems} level.
      \item Each \texttt{System} has a degree-of-freedom mapping
        \texttt{DofMap} that assigns any DoF indices to each
        \texttt{DofObject} (\texttt{Elem} or \texttt{Node})
    \end{itemize}
  \end{block}
\end{frame}

\begin{frame}[shrink]
  \frametitle{EquationSystems}
  \lstinputlisting{snippets/es.cxx}
\end{frame}




%%%%%%%%%%%%%%%%%%%%%%%%%%%%%%%%%%%%%
\subsection{The \texttt{Elem} and \texttt{Node} Classes}
\begin{frame}
  \frametitle{Elements and Nodes}
  \begin{block}{Elements}
    \begin{itemize}
      \item The \texttt{Elem} base class is the interface to any geometric element in \libMesh{}.
      \item An \texttt{Elem} is defined by \texttt{Node}s, Edges (2D, 3D) and Faces (3D).
      \item An \texttt{Elem} is also a \texttt{DofObject} with
        more metadata:
        \begin{itemize}
          \item Global ID.
          \item Processor ownership.
          \item Degree of freedom indexing data.
          \item ``Extra'' integers or data
        \end{itemize}
    \end{itemize}
  \end{block}
  \begin{block}{Nodes}
    \begin{itemize}
      \item A \texttt{Node} is a \texttt{DofObject} which is also a
        \texttt{Point}, defining a location in 1D/2D/3D
    \end{itemize}
  \end{block}

\end{frame}

\begin{frame}[shrink]
  \frametitle{Elements and Nodes}
  \lstinputlisting{snippets/elem.cxx}
\end{frame}

