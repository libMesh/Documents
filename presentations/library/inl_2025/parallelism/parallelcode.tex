
\subsubsection{Parallel Programming}
%%%%%%%%%%%%%%%%%%%%%%%%%%%%%%%%%%%%%%%%%%%%%%%%%%%%%%%%%%%%%%%%%%%%%
%\royslide{Parallel:: API}{
\begin{frame}[fragile]
\frametitle{TIMPI: Templated Interface to MPI}
\royitemizebegin{Encapsulating message passing}
\item Improvement over MPI C++ interface
\item Makes code much shorter, more legible
\item Enables faster backend implementations
\royitemizeend

Example:
\small
\begin{lstlisting}
std::vector<Real> send, recv;
...
send_receive(dest_processor_id, send,
             source_processor_id, recv);
\end{lstlisting}
\end{frame}
%}



%%%%%%%%%%%%%%%%%%%%%%%%%%%%%%%%%%%%%%%%%%%%%%%%%%%%%%%%%%%%%%%%%%%%%
%\royslide{Parallel:: API}{
\begin{frame}[fragile,shrink]
\frametitle{TIMPI: Templated Interface to MPI}

Instead of:
\begin{lstlisting}
if (dest_processor_id   == libMesh::processor_id() &&
    source_processor_id == libMesh::processor_id())
  recv = send;
#ifdef HAVE_MPI
else
  {
    unsigned int sendsize = send.size(), recvsize;
    MPI_Status status;
    MPI_Sendrecv(&sendsize, 1, datatype<unsigned int>(),
                 dest_processor_id, 0,
                 &recvsize, 1, datatype<unsigned int>(),
                 source_processor_id, 0,
                 libMesh::COMM_WORLD,
                 &status);

    recv.resize(recvsize);

    MPI_Sendrecv(sendsize ? &send[0] : NULL, sendsize, MPI_DOUBLE,
                 dest_processor_id, 0,
                 recvsize ? &recv[0] : NULL, recvsize, MPI_DOUBLE,
                 source_processor_id, 0,
                 libMesh::COMM_WORLD,
                 &status);
  }
#endif // HAVE_MPI
\end{lstlisting}
\end{frame}
%}


\begin{frame}[fragile]
\frametitle{TIMPI Templating}
\royitemizebegin{TIMPI method input types}
\item Templating + recursion enables arbitrary input data
\item \texttt{StandardType<T>} handles any fixed-size \texttt{T}
\item \texttt{Packing<T>} handles any variable-size \texttt{T}
\item Specializations for \texttt{T<U,V>} operate recursively
\royitemizeend

\pause

Example:
\small
\begin{lstlisting}
std::vector<std::pair<std::set<unsigned int>, unsigned int>> vals;
const unsigned int my_rank = TestCommWorld->rank();

TestCommWorld->allgather(std::make_pair(
                           createSet<unsigned int>(my_rank + 1),
                           my_rank), vals);
\end{lstlisting}

\pause

\texttt{StandardType<T>} and \texttt{Packing<T>} are predefined for
  standard types, defined in separate headers for \libMesh{} and
  \texttt{MetaPhysicL} types, and specializable for user types.


\end{frame}



\begin{frame}[fragile]
\frametitle{TIMPI Parallel Synchronization Algorithms}
\royitemizebegin{Pushing or Pulling Data}
\item We have data to send to a sparse set of processors
  \begin{itemize}
    \item Queries, if only the receiver knows what it needs
  \end{itemize}
\item We need to act on that data when we get it
\royitemizeend

\pause

Example:
\small
\begin{lstlisting}
std::map<processor_id_type, std::vector<unsigned int> > send, receive;

fill_scalar_data(send);

auto collect_data =
  [&received_data]
  (processor_id_type pid,
   const std::vector<unsigned int> & vec_received)
  {
    auto & vec = receive[pid];
    vec.insert(vec.end(), vec_received.begin(), vec_received.end());
  };

TIMPI::push_parallel_vector_data(*TestCommWorld, send, collect_data);
\end{lstlisting}

\end{frame}

