\subsection{Parallelism}

%%%%%%%%%%%%%%%%%%%%%%%%%%%%%%%%%%%%%%%%%%%%%%%%%%%%%%%%%%%%%%%%%%%%%
\royslide{Parallelism Goals}{
\royitemizebegin{Reduced CPU time}
\item Distributed CPU usage
\item Asynchronous I/O
\royitemizeend

\royitemizebegin{Reduced memory requirements}
\item Larger attainable problem size
\royitemizeend
}



%%%%%%%%%%%%%%%%%%%%%%%%%%%%%%%%%%%%%%%%%%%%%%%%%%%%%%%%%%%%%%%%%%%%%
\royslide{Parallelism Levels}{

\begin{columns}
\begin{column}{0.7\textwidth}
\royitemizebegin{Single Instruction, Multiple Data Operations}
\item ``Easy'': CPU SIMD in element assemblies, matvec operations,
  etc.
\pause
\item Hard: GPU SIMD across multiple elements, with heterogeneous
  element and/or FE type
\royitemizeend

\royitemizebegin{Shared-memory Multiple Instruction, Multiple Data Operations}
\item Via \texttt{Threads::} range splitting in \libMesh{}
\item Easier than distributed-memory algorithms?
  Each thread can access all the data it needs.
\pause
\item Much harder than message-passing?
  Each thread has direct access to each other's data.
\royitemizeend
\end{column}

\begin{column}{0.25\textwidth}
  \begin{center}
    \includegraphics[width=.9\textwidth]{parallelism/SIMD}
  \end{center}
\end{column}
\end{columns}
}



%%%%%%%%%%%%%%%%%%%%%%%%%%%%%%%%%%%%%%%%%%%%%%%%%%%%%%%%%%%%%%%%%%%%%
\royslide{Parallelism Levels}{

\begin{columns}
\begin{column}{0.5\textwidth}
\royitemizebegin{Separate Simulations}
\item Parametric studies
\item Uncertainty analysis
\pause
\item ``Embarrassingly parallel''
\royitemizeend

\royitemizebegin{Distributed-memory Processes}
\item Asynchronous I/O
\item Mesh Partitioning
\royitemizeend
\end{column}

\begin{column}{0.5\textwidth}
  \begin{center}
    \includegraphics[width=.9\textwidth]{parallelism/Meshes}
  \end{center}
\end{column}
\end{columns}
}
