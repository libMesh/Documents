\section*{Outline}% Make it easy to jump to this page in the PDF

% use outline_currentsection.tex to highlight the current section

% Auto-generate the TOC slide(s)
\begin{frame}
  %\tableofcontents[currentsection]
  \tableofcontents
\end{frame}






% The optional argument [<+->] means everything on the frame will be displayed incrementally.
\section{Introduction}
% Auto-generate the TOC slide(s)
\begin{frame}
  \tableofcontents[currentsection]
  %\tableofcontents
\end{frame}


%%%%%%%%%%%%%%%%%%%%%%%%%%%%%%%%%%%%%%%%%%%%%%%%5
\frame
{
  \frametitle{Background}                 

  \begin{itemize}
  \item Modern simulation software is \textcolor{nasablue}{complex}:
    \begin{itemize}
    \item Implicit numerical methods
    \item Massively parallel computers
    \item Adaptive methods
    \item Multiple, coupled physical processes
    \end{itemize}
    %\pause
  \item There are a host of existing software libraries that excel at treating various aspects of this complexity.
  \item Leveraging existing software whenever possible is the most efficient way to manage this complexity.

  \end{itemize}
}


 

%%%%%%%%%%%%%%%%%%%%%%%%%%%%%%%%%%%%%%%%%%%%%%%%5
\frame
{
  \frametitle{Background}                 

  \begin{itemize}
  \item Modern simulation software is \textcolor{nasablue}{multidisciplinary}:
    \begin{itemize}
    \item Physical Sciences
    \item Engineering
    \item Computer Science
    \item Applied Mathematics
    \item \ldots
    \end{itemize}
  \item It is not reasonable to expect a single person to have all the necessary skills for developing \& implementing high-performance numerical algorithms on modern computing architectures.
  \item Teaming is a prerequisite for success.
  \end{itemize}
}


 

%%%%%%%%%%%%%%%%%%%%%%%%%%%%%%%%%%%%%%%%%%%%%%%%5
\frame
{
  \frametitle{Background}                 
  \begin{itemize}
    \item A large class of problems are amenable to \textcolor{nasablue}{mesh based} simulation techniques.
      %% \begin{columns}[t]
      %%   \column{.5\textwidth}        
      %%   \fbox{\includegraphics[viewport=140 420 400 685,clip=true,height=1in]{domain2/domain2_input}}
      %%   \column{.5\textwidth}
      %%   \fbox{\includegraphics[height=1in,angle=-90]{discretized_domain}}
      %% \end{columns}
    \item Consider some of the major components such a simulation:
      \pause
      \begin{enumerate}
        \item Read the mesh from file
        \item Initialize data structures
        \item Construct a discrete representation of the governing equations
        \item Solve the discrete system
        \item Write out results
        \item Optionally estimate error, refine the mesh, and repeat
      \end{enumerate}

    \pause
    \item With the exception of step 3, the rest is \emph{independent} of the class of problems being solved.
    \pause
    \item This allows the major components of such a simulation to be abstracted \& implemented in a reusable software library.
  \end{itemize}
}


 

%%%%%%%%%%%%%%%%%%%%%%%%%%%%%%%%%%%%%%%%%%%%%%%%5
\frame
{
  \frametitle{The \libmesh{} Software Library}
  \begin{itemize}
    \item In 2002, the \libmesh{} library began with these ideas in mind.
    \item Primary goal is to provide data structures and algorithms that can be shared by disparate physical applications, that may need some combination of
      \begin{itemize}
      \item Implicit numerical methods
      \item Adaptive mesh refinement techniques
      \item Parallel computing
      \end{itemize}
    \item Unifying theme: \textcolor{nasablue}{mesh-based simulation of partial differential equations}.
  \end{itemize}
}



 

%%%%%%%%%%%%%%%%%%%%%%%%%%%%%%%%%%%%%%%%%%%%%%%%5
\frame
{
  \frametitle{The \libmesh{} Software Library}

  \begin{block}{Key Point}
    \begin{itemize}
      \item The \libmesh{} library is designed to be used by researchers, scientists, and engineers \textcolor{nasablue}{as a tool for developing simulation codes}.
      \item \libMesh{} is not an application code.
      \item It does not ``solve problem XYZ.''
        \begin{itemize}
          \item It can be used to help you develop an application to solve problem XYZ, and to do so quickly with advanced numerical algorithms on high-performance computing platforms.
        \end{itemize}
      %\item It was initially targeted for finite element based simulations, but has been used for finite volume discretizations as well.
    \end{itemize}    
  \end{block}
} 



%%%%%%%%%%%%%%%%%%%%%%%%%%%%%%%%%%%%%%%%%%%%%%%%5
\frame
{
  \frametitle{Software Reusability}
}

 
\subsection*{Library Description}
\begin{frame}[t]
  %\frametitle{LibMesh Tree}
%  \vspace{-.25in}
%  \begin{center}
%    \includegraphics[width=.6\textwidth]{mytreeandroots_allnames}    
%  \end{center}


    \begin{minipage}[h]{.6\textwidth}
    \begin{center}
      \includegraphics[width=.9\textwidth]{mytreeandroots_allnames}
    \end{center}
  \end{minipage}
  \begin{minipage}[h]{.35\textwidth}
    \begin{block}{Library Structure}
      \begin{itemize}
        %\small
    \item Basic libraries are \libMesh{}'s ``roots''
    \item Application ``branches'' built off the library ``trunk''
      \end{itemize}
    \end{block}
  \end{minipage}



\end{frame}

\subsection*{What class of problems is LibMesh designed to solve?}
% The ``Generic BVP'' slide has been slightly revamped for notational consistency
\begin{frame}[t]
  %\frametitle{A Generic BVP}
  \begin{columns}[t]
    \column{.5\textwidth}
     \begin{itemize}
      \item General boundary value problems of the form:
      \end{itemize}
%    \begin{block}{}%{We assume there is a Boundary Value Problem of the form}
      \vspace{-.1in}
        \begin{eqnarray}
	\label{eqn:general_pde}
	\nonumber
	M \frac{\partial u}{\partial t} & = & F( u ) \;\;\;\; \in \Omega
        \\
	\nonumber
	G( u ) & = & 0 \;\;\;\;\;\;\;\;\; \in \Omega
	\\
	\nonumber
	u & = & u_D \;\;\;\;\;\;\; \in \partial \Omega_D
	\\
	\nonumber
	N(u) & = & 0 \;\;\;\;\;\;\;\;\; \in \partial \Omega_N
 	\\
 	\nonumber
 	u(\bv{x}, 0) & = & u_0(\bv{x}) 
      \end{eqnarray}
%    \end{block}
    %\pause
    \column{.5\textwidth}
      \begin{center}
	\includegraphics[viewport=140 420 400 685,clip=true,width=2in]{figures/domain2/domain2_input}
      \end{center}
  \end{columns}
\end{frame}

\begin{frame}
  %\frametitle{}
  \begin{columns}[t]
    \column{.5\textwidth}
    \begin{block}{}%A general class of PDE}
      \begin{itemize}
      \item{
	Associated to the problem domain $\Omega$ is a \libMesh{} data
	structure called a \texttt{Mesh}
      }

      \item{A \texttt{Mesh} is essentially
	a collection of finite elements}
      \end{itemize}
      \begin{equation}
	\label{eqn:discretized_domain}
	\nonumber
	\Omega^h:=\bigcup_e \Omega_e
      \end{equation}
    \end{block}
    %\pause
    \column{.5\textwidth}
    %\begin{block}{}
      \begin{center}
	%\fbox{
	\includegraphics[width=2in,angle=-90]{discretized_domain}
	%}
      \end{center}
    %\end{block}
  \end{columns}
  \visible<2>
  {
  \begin{itemize}
    \item{\libMesh{} provides some simple structured mesh generation
routines, interfaces to Triangle and TetGen, and supports a rich set of input file formats.}
  \end{itemize}
  }
\end{frame}




% LocalWords:  nasablue
