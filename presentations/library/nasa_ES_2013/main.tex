\documentclass[
  compress
  ,12pt
  %,draft
  %,handout % I don't see how this could ever work for my presentation!
]{beamer}



%%%%%%%%%%%%%%%%%%%%%%%%%%%
% LaTeX package inclusion %
%%%%%%%%%%%%%%%%%%%%%%%%%%%
\usepackage{times}
\usepackage{units}
\usepackage{mathrsfs}
\usepackage{diss} % defines \bv and some other stuff
\usepackage{subfigure}
\usepackage{multirow}
\usepackage{amsmath}
\usepackage{amssymb}
\usepackage{multimedia}
%\usepackage{algorithm}
%\usepackage{algorithmic}
\usepackage{hyperref}
\usepackage{multimedia}




 \usefonttheme[
   onlymath
 ]{serif} %onlymath option doesn't look too bad, eqns are more readable this way.


\usetheme{Darmstadt}
\usecolortheme{orchid} % white on dark block titles.  use w/ whale.
\usecolortheme{whale} % darkest top titles, usually used by CFDLab presenters
\setbeamertemplate{itemize items}[circle]% Force any theme (eg Antibes) to use circle bullets
%\useinnertheme[shadow]{rounded} % Causes itemize blocks to have rounded corners & drop shadows
\logo{\includegraphics[width=.5in]{figures/word3}}
%\setbeamercolor{title}{fg=red!80!black,bg=red!20!white} %pink title!

\title{LibMesh Experience and Usage}
\author[Kirk, Peterson, Stogner]{Benjamin S.\ Kirk \\ \texttt{\scriptsize benjamin.kirk@nasa.gov} \\
  John W.\ Peterson \\ \texttt{\scriptsize peterson@cfdlab.ae.utexas.edu} \\
  Roy H.\ Stogner \\ \texttt{\scriptsize roystgnr@cfdlab.ae.utexas.edu}}
\institute[NASA, INL, UT]{NASA Lyndon B. Johnson Space Center \\ Idaho National Labs \\ The University of Texas at Austin}
\date{\today}

%\AtBeginSection[]{\frame{\tableofcontents[current]}}

% Dr. Carey's Notes:
% I had in mind some of the material we have in the Libmesh papers well 
% as  from the perspective of:(1) a very knowlegable  developer,(2) 
% installing an application with experience (3)  applying an 
% application for parametric runs  with some examples. perhaps some of 
% the issues encounter, as in all such codes like stopping criteria for 
% adaptive refinement, selection of error indicators, flexibility in 
% model adaption, ( adding terms, modifying constitutive models etc). 
% perhaps something on the coarsening /refining ; one level exceptions 
% etc. it would be helpful if you, Ben and Roy coordinated this. 
% Someone should talk about PETSc, partitioning etc. Someone about 
% memory restrictions etc. Strengths and limitations.  examples, couple 
% of movies etc etc


% Abstract:
% This talk will focus on several of the practical aspects involved in
% using the LibMesh library for finite element analysis.  The topics
% covered will include: the steps in going from a mathematical model
% (PDE) to a working implementation (code), stopping criteria for
% adaptive refinement, the selection of error indicators, and model
% adaptation (adding terms, changing constitutive laws, etc).  The
% strengths, weaknesses, and current limitations of the library will
% be discussed in the same practical context.  Finally, some additional
% examples giving a flavor of the types of applications which have
% already been developed around the library will be given.
\setbeamercovered{transparent}


\newcommand{\R}{\mathscr{R}}
\newcommand{\LibMesh}{\texttt{Lib\-Mesh}}

\begin{document}


  
\begin{frame}
  \titlepage
\end{frame}



%=================================================================
% Outline
%=================================================================
%\section{Introduction}
%% Auto-generate the TOC slide(s)
\begin{frame}
  \tableofcontents[currentsection]
  %\tableofcontents
\end{frame}

\section*{Outline}% Make it easy to jump to this page in the PDF

% use outline_currentsection.tex to highlight the current section

% Auto-generate the TOC slide(s)
\begin{frame}
  %\tableofcontents[currentsection]
  \tableofcontents
\end{frame}





\section{\libMesh{} History and Goals}
%%%%%%%%%%%%%%%%%%%%%%%%%%%%%%%%%%%%%%%%%%%%%%%%%
\frame
{
  \frametitle{History}

  \begin{center}
  \includegraphics[width=.3\textwidth]{cfdlab}
  \end{center}

  \begin{itemize}
    \item CFDLab, University of Texas at Austin
      \begin{itemize}
        \item Computational Fluid Dynamics research, from creeping
          (Roy Stogner) to incompressible (John Peterson) to hypersonic
          (Benjamin Kirk)
        \item Adaptive Finite Element Method research (Dr. Graham F.\ Carey)
        \item High-performance computing research (Bill Barth)
        \item Software engineering experience (Robert McLay)
        \item<3-> Stubbornness (Benjamin Kirk, John Peterson, Roy Stogner, ...)
      \end{itemize}
    \item<2-> ``No one ever got a PhD from here for writing a code.'' - Dr. Graham F.\ Carey
  \end{itemize}  


}



%%%%%%%%%%%%%%%%%%%%%%%%%%%%%%%%%%%%%%%%%%%%%%%%%
\frame
{
  \begin{center}
  \includegraphics[width=.3\textwidth]{cfdlab}
  \end{center}

  \begin{block}{The standard Ph.D.-candidate software development
    process:}
  \pause
  \begin{itemize}[<+->]
    \item Create pseudocode
      \begin{itemize}[<+->]
        \item (throw it away; it won't match the real code you
          end up writing)
      \end{itemize}
    \item Create Matlab prototypes, fast
      \begin{itemize}[<+->]
        \item You're just solving one problem
        \item Don't waste time on
          documentation for slow-witted users
        \item You're just going to rewrite the bad hacks
          anyway
      \end{itemize}
    \item Rewrite from scratch (maybe in Fortran?)
      \begin{itemize}[<+->]
        \item ``the single worst strategic mistake'' - Joel Spolsky, 2000
        \item ``Other slow-witted users'' includes ``myself, 2 years
          later''
        \item Bad hacks may now be load-bearing dependencies
      \end{itemize}
    \item Graduate, throw it all away, go work on a ``real'' code
  \end{itemize}  
  \end{block}


}


 

%%%%%%%%%%%%%%%%%%%%%%%%%%%%%%%%%%%%%%%%%%%%%%%%%
\frame
{
  \begin{center}
  \includegraphics[width=.3\textwidth]{cfdlab}
  \end{center}

  \begin{block}{libMesh Development Ingredients}
  \pause
  \begin{itemize}[<+->]
    \item Inspiration (Wolfgang Bangerth + deal.II, Robert McLay + MGFLO)
    \item Genius (Ben Kirk, John Peterson, 2002)
    \item Collaboration (Daniel Dreyer, Steffen Petersen, 2003; Roy
      Stogner 2004; David Knezevic 2005; Derek Gaston 2006)
    \item Flexibility
      \begin{itemize}[<+->]
        \item Documentation: teaching is the best way to learn
        \item Abstraction: purpose first, algorithm second, data third
          \begin{itemize}[<+->]
            \item C++: OOP \emph{and} ``only pay for what you use''
          \end{itemize}
        \item Encapsulation: only promise what you can guarantee
        \item Modularity: only use what you must demand
        \item Testing: don't break what you promise
      \end{itemize}
  \end{itemize}  
  \end{block}


}


 




%%%%%%%%%%%%%%%%%%%%%%%%%%%%%%%%%%%%%%%%%%%%%%%%%
\frame
{
  \frametitle{The \libMesh{} Software Library}
  \begin{itemize}
    \item In 2002, the \libMesh{} library was created with these ideas
      and concerns in mind.
    \item Primary goal is to provide individual tools: data structures
      and algorithms that can be shared by widely differing physical
      applications, that may need some combination of
      \begin{itemize}
      \item Implicit numerical methods
      \item Adaptive mesh refinement techniques
      \item Parallel computing 
      \end{itemize}
    \item Unifying theme: \emphcolor{mesh-based simulation of partial differential equations (PDEs)}.
      \begin{itemize}
      \item Continuous and discontinuous Finite Element Methods
      \item Finite Volume Methods
      \item Even many ``mesh-free'' methods
      \end{itemize}
  \end{itemize}
}




 

%%%%%%%%%%%%%%%%%%%%%%%%%%%%%%%%%%%%%%%%%%%%%%%%%
\frame
{
  \frametitle{The \libMesh{} Software Library}

  \begin{block}{A Toolkit, not a Framework}
    \begin{itemize}
      \item \libMesh{} was designed to give students, researchers,
        scientists, and engineers tools to \emphcolor{develop
        simulation codes} or \emphcolor{rapidly implement a numerical
        method}.
      \item \libMesh{} is not an application.
      \item It does not ``solve problem XYZ.''
        \begin{itemize}
          \item It was designed to help \cancel{users} researchers
            create their own application to solve problem XYZ,
            quickly, with advanced numerical algorithms on
            high-performance computing platforms.
          \item It has since been used to create professional
            application frameworks to solve many problems, combined,
            with user-friendly interfaces.
        \end{itemize}
    \end{itemize}    
  \end{block}
} 


%%%%%%%%%%%%%%%%%%%%%%%%%%%%%%%%%%%%%%%%%%%%%%%%%
\frame
{
  \frametitle{Software Reuse}
  \begin{itemize}
    \item When \libMesh{} was created in 2002, many high-quality
      software libraries implemented some fraction of the end-to-end PDE simulation process:
      \begin{itemize}
        \item Linear algebra (Laspack, PETSc)
        \item Partitioning algorithms for domain decomposition
        \item Visualization of solution files
        \item \ldots
      \end{itemize}
    \item \libMesh{} tries to provide flexible, extensible, abstract interfaces to existing software when possible.
    \item We implemented ``glue'' to these pieces, plus what was missing at the time:
      \begin{itemize}
        \item \emphcolor{Flexible data structures for discretizating of spatial domains and systems of PDEs posed on them.}
      \end{itemize}          
  \end{itemize}  
}




\begin{frame}{libMesh Community}
\begin{columns}
\column{.4\textwidth}
\begin{block}{Scope}
\begin{itemize}
\item Free, Open source
\begin{itemize}
\item LGPL2 for core
\end{itemize}
\item 153 Ph.D.\ theses from users, 2254 papers (240+ in 2024)
\item 15 contributors in 2024, 102 historically
\end{itemize}
\end{block}

\column{.6\textwidth}
\includegraphics[width=.45\textwidth]{ablating_hs_wbg}
\includegraphics[width=.25\textwidth]{sov}
\includegraphics[width=.25\textwidth]{marmot1b}
\end{columns}

\begin{columns}
\column{.35\textwidth}
\includegraphics[width=\textwidth]{cloc_libmesh}

\column{.65\textwidth}
\begin{block}{Challenges}
\begin{itemize}
\item Radically different application types
\item Widely dispersed core developers
\begin{itemize}
\item At peak: INL, UT-Austin, U.Buffalo, JSC, MIT, Harvard, Argonne
\end{itemize}
\item OSS, commercial, private applications
\end{itemize}
\end{block}
\end{columns}

\end{frame}


\begin{frame}[t]
  \begin{columns}
    \column{.6\textwidth}
    \begin{center}
      \includegraphics[width=.9\textwidth]{mytreeandroots_allnames}
    \end{center}
    \column{.35\textwidth}
    \begin{itemize}
      \item Foundational (typically optional) library access via LibMesh's ``roots''.
      \item Application ``branches'' built off the library ``trunk''.
      \item Additional middleware layers (e.g. Akselos, GRINS, MOOSE) for more complex applications
    \end{itemize}
  \end{columns}

\end{frame}

\subsection*{Non-Trivial Applications}
\begin{frame}%[t]
%  \frametitle{Weighted Residual Connection}
  %\begin{block}{}
  \begin{itemize}%[<+->]
  \item{\libMesh{} provides several of the tools necessary to construct
    these systems, but it is not specifically written to solve any one
    problem.}

  \item{First, a few of the non-trivial applications which have been built on
    top of the library.}
    
%%   \item{In each case, the matrix $A$ is the ``Jacobian''
%%     operator, and the right-hand side vector $b$ is the
%%     weighted residual itself.}

    %% 	\item{This is true even in the case of linear $\R( u )$, since
    %% 	  in this case the linearized operator is
    %% 	  \begin{eqnarray}
    %% 	    \nonumber
    %% 	    \R'( u )w &:=& \lim_{\varepsilon\rightarrow 0}
    %% 	    \frac{\R(u+\varepsilon w) - \R(u)}{\varepsilon} \\
    %% 	    \nonumber
    %% 	    &=& \R(w)
    %% 	  \end{eqnarray}
    %% 	}
  \end{itemize}
%\end{block}
\end{frame}	  

\subsection*{Natural Convection}
\begin{frame}[t]
  \begin{center}
    \includegraphics[width=.45\textwidth]{figures/part_trans}
    %\\
    \includegraphics[width=.45\textwidth]{figures/streamtraces}
  \end{center}
  \begin{block}{}
    \begin{itemize}
    \item{
      Tetrahedral mesh of ``pipe'' geometry.
      Stream ribbons colored by temperature.
      }
      \end{itemize}
  \end{block}
\end{frame}

\subsection*{Surface-Tension-Driven Flow}
\begin{frame}[t]
  \begin{center}
    \includegraphics[width=.6\textwidth]{rbm_adapt_soln}    
  \end{center}

  \begin{block}{}
    \begin{itemize}
    \item{Adaptive grid solution shown
      with temperature contours and velocity vectors.
      }
      \end{itemize}
  \end{block}
\end{frame}

\subsection*{Double-Diffusive Convection}
\begin{frame}[t]
  \begin{center}
    \includegraphics[width=.6\textwidth]{dd}    
  \end{center}

  \begin{block}{}
    \begin{itemize}
    \item{Solute contours: a plume of
      warm, low-salinity fluid is convected upward through a porous medium.
      }
      \end{itemize}
  \end{block}
\end{frame}



\subsection*{Tumor Angiogenesis}
\begin{frame}[t]
  \begin{center}
    \includegraphics[width=.6\textwidth]{figures/tumor_model}    
  \end{center}

  \begin{block}{}
    \begin{itemize}
    \item{%Tumor angiogenesis model simulation.
      The tumor secretes
      a chemical which stimulates blood vessel formation.
      }
      \end{itemize}
  \end{block}
\end{frame}



\subsection*{Phase Separation}
\begin{frame}[t]
  \begin{center}
    \includegraphics[width=.3\textwidth]{ch3D02-006}    
    \includegraphics[width=.3\textwidth]{ch3D02-024}    
    \includegraphics[width=.3\textwidth]{ch3D02-096}    
  \end{center}

  \begin{block}{}
    \begin{itemize}
    \item{%Tumor angiogenesis model simulation.
      Directed pattern self-assembly in spinodal decomposition of binary
mixture
      }
      \end{itemize}
  \end{block}
\end{frame}



\subsection*{Compressible Flow}

\frame
{
  \frametitle{Compressible Shocked Flow}
  \begin{itemize}[<+->]
    \item Original compressible flow code written by Ben Kirk utilizing \libMesh{}.
      \begin{itemize}[<+->]
      \item Solves both Compressible Navier Stokes and Inviscid Euler.
      \item Includes both SUPG and a shock capturing scheme.
      \end{itemize}
    \item Original redistribution code written by Larisa Branets.
      \begin{itemize}[<+->]
      \item Simultaneous optimization of element shape and size.
      \item Directable via user supplied error estimate.
      \end{itemize}
    \item Integration work done by Derek Gaston.
      \begin{itemize}[<+->]
      \item Combination of redistribution, $h$ refinement.
      \item Applicable to other problem classes.
      \end{itemize}
  \end{itemize}
}

\frame
{
  \frametitle{Problem Specification}
  \begin{itemize}[<+->]
    \item The problem studied is that of an oblique shock generated by a $10^o$ wedge angle. 
      \begin{itemize}[<+->]
      \item This problem has an exact solution for density which is a step function.
      \item Utilizing \libMesh{}'s exact solution capability the exact
$L_2$ error can be solved for.
      \item The exact solution is shown below:
        \begin{figure}
          \begin{center}
            \includegraphics[viewport=20 10 660 600,clip=true,width=.4\textwidth]{shock.pdf}
          \end{center}
        \end{figure}
    \end{itemize}
  \end{itemize}
}

\frame
{
  \frametitle{Uniformly Refined Solutions}
  \begin{itemize}[<+->]
  \item For comparison purposes, here is a mesh and a solution after 1 uniform refinement with 10890 DOFs.
    \begin{figure}[!htb]
      \begin{center}
        \subfigure[Mesh after 1 uniform refinement.]{\label{fig:fob_uniform_2_mesh}\includegraphics[viewport=110 30 600 550,clip=true,width=.42\textwidth]{fob_uniform_2_mesh.pdf}}
        \subfigure[Solution after 1 uniform refinement.]{\label{fig:fob_uniform_2_sol}\includegraphics[viewport=110 30 600 520,clip=true,width=.42\textwidth]{fob_uniform_2_sol.pdf}}
      \end{center}
    \end{figure}
  \end{itemize}
}

\frame
{
  \frametitle{H-Adapted Solutions}
  \begin{itemize}[<+->]
    \item A flux jump indicator was employed as the error indcator along with a statistical flagging scheme.
    \item Here is a mesh and solution after 2 adaptive refinements containing 10800 DOFs:
      \begin{figure}[!htb]
        \begin{center}
          \subfigure[Mesh, 2 refinements]{\label{fig:fob_adapt_3_mesh}\includegraphics[viewport=110 30 600 550,clip=true,width=.42\textwidth]{fob_adapt_3_mesh.pdf}}
          \subfigure[Solution]{\label{fig:fob_adapt_3_sol}\includegraphics[viewport=110 30 600 520,clip=true,width=.42\textwidth]{fob_adapt_3_sol.pdf}}
        \end{center}
      \end{figure}
  \end{itemize}
}

\frame
{
  \frametitle{Redistributed Solutions}
  \begin{itemize}[<+->]
    \item Redistribution utilizing the same flux jump indicator.
      \begin{figure}[!htb]
        \begin{center}
          \subfigure[Mesh, 8 redistribution steps]{\label{fig:fob_redist_adapt_8_mesh}\includegraphics[viewport=110 30 600 550,clip=true,width=.42\textwidth]{fob_redist_adapt_8_mesh.pdf}}
          \subfigure[Solution]{\label{fig:fob_redist_adapt_8_sol}\includegraphics[viewport=110 30 600 520,clip=true,width=.42\textwidth]{fob_redist_adapt_8_sol.pdf}}
        \end{center}
      \end{figure}
  \end{itemize}
}

\frame
{
  \frametitle{Redistributed and Adapted}
  \begin{itemize}[<+->]
    \item Now combining the two, here are the mesh and solution after 2 adaptations beyond the previous redistribution containing 10190 DOFs.
      \begin{figure}[!htb]
        \begin{center}
          \subfigure[Mesh, 2 refinements]{\label{fig:fob_redist_adapt_10_mesh}\includegraphics[viewport=110 30 600 550,clip=true,width=.42\textwidth]{fob_redist_adapt_10_mesh.pdf}}
          \subfigure[Solution]{\label{fig:fob_redist_adapt_10_sol}\includegraphics[viewport=110 30 600 520,clip=true,width=.42\textwidth]{fob_redist_adapt_10_sol.pdf}}
        \end{center}
      \end{figure}
  \end{itemize}
}

\frame
{
  \frametitle{Solution Comparison}
  \begin{itemize}[<+->]
    \item For a better comparison here are 3 of the solutions, each with around 11000 DOFs:
      \begin{figure}[!htb]
        \begin{center}
          \subfigure[Uniform.]{\label{fig:fob_uniform_2_sol}\includegraphics[viewport=110 30 600 520,clip=true,width=.3\textwidth]{fob_uniform_2_sol.pdf}}
          \subfigure[Adaptive.]{\label{fig:fob_adapt_3_sol}\includegraphics[viewport=110 30 600 520,clip=true,width=.3\textwidth]{fob_adapt_3_sol.pdf}}
          \subfigure[R + H.]{\label{fig:fob_redist_adapt_10_sol}\includegraphics[viewport=110 30 600 520,clip=true,width=.3\textwidth]{fob_redist_adapt_10_sol.pdf}}
        \end{center}
      \end{figure}
  \end{itemize}
}

\frame
{
  \frametitle{Error Plot}
  \begin{itemize}[<+->]
    \item \libMesh{} provides capability for computing error norms against an exact solution.
    \item The exact solution is not in $H^1$ therefore we only obtain
the $L_2$ convergence plot:
      \begin{figure}[!htb]
      \begin{center}
        \subfigure[LogLog plot of L2 vs DOFs.]{\label{fig:fob_l2}\includegraphics[viewport=0 10 600 400,clip=true,width=.7\textwidth]{fob_l2.pdf}}
      \end{center}
      \end{figure}
  \end{itemize}
}

\frame
{
  %\frametitle{Other Compressible Flow Examples}
    \begin{figure}[!htb]
      \begin{center}
        \subfigure{\label{fig:fob_uniform_2_sol}\includegraphics[width=.4\textwidth]{Hypersonic_cow_mach}}
        \subfigure{\label{fig:fob_adapt_3_sol}\includegraphics[width=.4\textwidth]{Benkirk_orbiter_reentry_side_view}}
        \subfigure{\label{fig:fob_redist_adapt_10_sol}\includegraphics[width=.4\textwidth]{Benkirk_schlieren}}
        \subfigure{\label{fig:fob_redist_adapt_10_sol}\includegraphics[width=.4\textwidth]{Benkirk_double_cone_M}}
      \end{center}
    \end{figure}
}

\section{Weighted Residuals}
% Auto-generate the TOC slide(s)
\begin{frame}
  \tableofcontents[currentsection]
  %\tableofcontents
\end{frame}


\begin{frame}[<+->]
      %\frametitle{Weighted Residual Statement}
  \begin{itemize}
  \item {The point of departure in any FE analysis which uses \libMesh{} is
    the weighted residual statement
    %(sometimes referred to as simply ``the residual'' in
    %the documentation.)
    \begin{equation}
      \nonumber
      (F( u ), v) = 0 \hspace{.5in} \forall v \in \mathcal{V}
    \end{equation}
    }

  \item{ Or, more precisely, the weighted residual statement associated with the
    finite-dimensional space $\mathcal{V}^h \subset \mathcal{V}$
    \begin{equation}
      \nonumber
      (F( u^{\alert{h}} ), v^{\alert{h}}) = 0 \hspace{.5in} \forall v^{\alert{h}} \in \mathcal{V}^{\alert{h}}
  \end{equation}}

  \end{itemize}
\end{frame}


\subsection*{Some Examples}    
\begin{frame}[t]
  %\frametitle{Some Examples}
    \begin{block}{
	\only<1-2>{Poisson Equation}
	\only<3-4>{Linear Convection-Diffusion}
	\only<5-6>{Stokes Flow}
      }

      \only<1-2>
      {
	\begin{equation}
	      \nonumber
	      -\Delta u  = f
	      \hspace{.25in} \in \hspace{.1in} \Omega  
	    \end{equation}
      }
      
      \only<3-4>
	  {
	    \begin{equation}
	      \nonumber
	      %\frac{\partial u}{\partial t}
	      -k\Delta u + \bv{b} \cdot \nabla u = f
	      \hspace{.25in} \in \hspace{.1in} \Omega  
	    \end{equation}
	  }

      \only<5-6>
      {
	\begin{equation}
	    \begin{array}{rcl}
	      \nonumber
	      %\frac{\partial \bv{u}}{\partial t} +
	      %\left(\bv{u} \cdot \nabla\right) \bv{u} +
	      \nabla p - \nu \Delta \bv{u}  &=& \bv{f}
	        \\
	      \nonumber
	      \nabla \cdot \bv{u} &=& 0
	    \end{array}  \hspace{.25in}  \in \hspace{.1in} \Omega
	\end{equation}
      }

      
\end{block}
    %\pause

    \only<2,4,6>
    {
    \begin{block}{Weighted Residual Statement}
    }
      \only<2>
      {
      \begin{eqnarray}
	\nonumber
	(F( u ), v) := %\hspace{3in} \\  \nonumber
	\int_{\Omega}  \left[ \nabla u \cdot \nabla v - fv \right] dx \\ \nonumber
	+ \int_{\partial \Omega_N} \left(\nabla u \cdot \bv{n}\right) v \;ds
      \end{eqnarray}
%%       $^{\ast}$ We have employed the divergence theorem to obtain the weighted residual statement.
%%       In general this procedure gives rise to boundary terms which for simplicity we do not discuss
%%       in detail.
      }
      
    \only<4>
    {
      \begin{eqnarray}
	\nonumber
	(F( u ), v) := 
	\int_{\Omega} \left[
	  %\tfrac{\partial u}{\partial t}v  +
	  k\nabla u \cdot \nabla v + (\bv{b} \cdot \nabla u) v - fv \right] dx \\ \nonumber
	+ \int_{\partial \Omega_N} k\left(\nabla u \cdot \bv{n}\right) v \;ds
      \end{eqnarray}
    }

    \only<6>
    {
      \vspace{-.2in}
      \begin{eqnarray}
	\nonumber
	u := \left[\bv{u}, p\right]
	\hspace{.1in},\hspace{.1in}
	v := \left[\bv{v}, q\right]
      \end{eqnarray}
      \vspace{-.25in}
	\begin{eqnarray}
	  \nonumber
	(F( u ), v) := %\hspace{3in} \\ \nonumber
	\int_{\Omega} \left[
	  %\left( \tfrac{\partial \bv{u}}{\partial t}	  +
	  %\left( \bv{u} \cdot \nabla  \right)\bv{u}
	  %\right)
	  %\cdot \bv{v}
	- p\left(\nabla \cdot \bv{v}\right) 
	+ \nu \nabla \bv{u} \colon\!\! \nabla \bv{v} - \bv{f}\cdot \bv{v} \; \right. \\ \nonumber
	+ \left.\left( \nabla \cdot \bv{u} \right) q \right] dx
	+ \int_{\partial \Omega_N} \left(\nu \nabla \bv{u} -p\bv{I}\right)  \bv{n} \cdot \bv{v} \;ds %\hspace{1in}	
      \end{eqnarray}
    }
\only<2,4,6>
{
    \end{block}
 }     
\end{frame} 



%\subsection*{Approximate Problem}
\begin{frame}%[<+->]
  %\frametitle{Weighted Residual Statement}
  \begin{itemize}

    %%   \item{In each of the examples, the weighted residual statement is obtained by
    %%     multiplying the PDE by a test function $v$, integrating over the domain $\Omega$,
    %%     and applying the divergence theorem.}

    %%   \item{Since $v=0$ on $\partial \Omega_D$ (essential data) the boundary integrals
    %%     are over $\partial \Omega_N$ only.}

    %%   \item{There are simple and efficient techniques (e.g.\ penalty method) for
    %%     enforcing the Dirichlet conditions.}

  \item{To obtain the approximate problem, we simply
    replace $u \leftarrow u^h$, $v \leftarrow v^h$, and $\Omega \leftarrow \Omega^h$
    in the weighted residual
    statement.}
    
  \end{itemize}
\end{frame}

\section{Poisson Equation}
% Auto-generate the TOC slide(s)
\begin{frame}
  \tableofcontents[currentsection]
  %\tableofcontents
\end{frame}


\subsection*{BVP framework: Poisson equation}
\begin{frame}%[<+->]
  %\frametitle{Poisson Equation}
  \begin{itemize}
  \item {As a representative example, consider the weak form 
      arising from the Poisson equation,
    \begin{eqnarray}
      \nonumber
      (F( u ), v_i) := \hspace{2.5in} \\  \nonumber
      \int_{\Omega^h}  \left[ \nabla u \cdot \nabla v_i - fv_i \right] dx %\\ \nonumber
      %+ \int_{\partial \Omega^h_N} u_N v^h \;ds
      =0 \hspace{.5in} \forall v_i \in \mathcal{V}
    \end{eqnarray}
  }
  \end{itemize}
\end{frame}

\subsection*{Element Integrals}
\begin{frame}%[c]
%  \frametitle{Poisson Equation}
  \begin{itemize}    
  \item{
%%     \only<1>
%% 	{
	  The integral over $\Omega^h$ \ldots
%%	}
	  \visible<2->
	  {
	    is written as
	    a sum of integrals over the $\alert{N_e}$ finite elements: % $\Omega_e^h$
	  }
  }
  \end{itemize}
	  
  %\begin{block}{}
    \begin{eqnarray}
	\nonumber
	%(F( u^h ), v^h) &:=& %\hspace{3in} \\  \nonumber
	0 &=&
	\phantom{\sum_{e=1}^{N_e}}
	\int_{\Omega^h}  \left[ \nabla u^h \cdot \nabla v^h - fv^h \right] dx
	\hspace{.2in} \forall v^{h} \in \mathcal{V}^{h}
	\\ \nonumber
	\visible<2>
	    {
	&=&\alert{\sum_{e=1}^{N_e}}
	      \int_{\alert{\Omega_e}}
	      \left[ \nabla u^h \cdot \nabla v^h - fv^h \right] dx
	      \hspace{.2in} \forall v^{h} \in \mathcal{V}^{h}
	      \\ \nonumber
	    }
%% 	    \visible<3>
%% 		{
%% 	&=&\alert{\sum_{e=1}^{N_e}}
%% 	      \underbrace{\int_{\alert{\Omega_e}}
%% 	      \left[ \nabla u^h \cdot \nabla v^h - fv^h \right] dx}_{\text{We must compute this}}
%% 	      \hspace{.2in} \forall v^{h} \in \mathcal{V}^{h}
%% 		}
      \end{eqnarray}
    %\end{block}
%%     \begin{eqnarray}
%%       \nonumber
%%       (F( u^h ), v^h) &=& \int_{\Omega^h} (\ldots) \\
%%       \nonumber
%%       &=& \sum_{e=1}^{N_e} \int_{\Omega_e}(\ldots)\hspace{.25in} \forall v^{h} \in \mathcal{V}^{h}
%%     \end{eqnarray}
    
%  \item{The $v^h$ typically have support over only a small subset of the elements.}
\end{frame}

\subsection*{Finite Element Basis Functions}
\begin{frame}
  % \frametitle{Weighted Residual Statement}
    \begin{columns}[t]
    \column{.5\textwidth}
    \begin{block}{}
%%       \only<1>
%%       {
%% 	To node $i$ we associate a basis function $\psi_i$ such that for any $v^h \in \mathcal{V}^h$
%% 	we have
%% 	\begin{equation}
%% 	  \nonumber
%% 	  v^h = \sum_{i=1}^{N_n} c_i \psi_i
%% 	\end{equation}
%% 	for some constants $c_i$.
%%       }

%%       \only<2>
%%       {
%% 	\begin{itemize}
%% 	  \item{The $\psi_i$ are non-zero only over the elements adjacent to node $i$.}
%% 	  \item{For example, $\psi_i$ could be the linear ``hat'' function.
%% 	    %with value 1
%% 	    %at node $i$ and zero at all other nodes.
%% 	  }
%% 	\end{itemize}
%%       }

%%       \only<3->
%%       {
	\begin{itemize}
	  \item{An element integral will have contributions only
	    from the global basis functions corresponding to its nodes.}
	  \item{We call these local basis functions $\phi_i$, $0 \leq i \leq N_s$.}
	\end{itemize}
%%      }
    \end{block}

%%       \visible<3->
%%       {
	    \begin{equation}
	      \nonumber
	      \left. v^h \right|_{\Omega_e} = \sum_{i=1}^{N_s} c_i \phi_i
	    \end{equation}
%%      }
      \visible<2>
      {
	    \begin{equation}
	      \nonumber
	      \alert{\int_{\Omega_e}} v^h \;\alert{dx}
	      = \sum_{i=1}^{N_s} c_i \alert{\int_{\Omega_e}}\phi_i \;\alert{dx}
	    \end{equation}

      }
%}
%  \end{itemize}
    \column{.5\textwidth}
    %\begin{block}{}
      \begin{center}
%% 	\only<1>
%% 	    {
%% 	      \includegraphics[width=2in,angle=-90]{node_i}
%% 	    }
%% 	\only<2>
%% 	    {
%% 	      \includegraphics[width=2in,angle=-90]{phi_i}
%% 	    }
%% 	\only<3->
%% 	    {
	      \includegraphics[width=2in,angle=-90]{phi_ijk}
%%	    }
      \end{center}
    \end{columns}
\end{frame}
    
\subsection*{Element Matrix and Load Vector}
\begin{frame}%[t]
%  \frametitle{Poisson Equation}
  \begin{itemize}    
    \visible<1->
	{
	\item
	  {
	    The element integrals \ldots
	    \begin{equation}
	      \nonumber
	      \int_{\Omega_e} \left[ \nabla u^h \cdot \nabla v^h - fv^h \right] dx
	    \end{equation}
	  }
	}

	
      \visible<2->
      {
	\item{
	  are written in terms of the local ``$\alert<2>{\phi_i}$'' basis functions
	  \begin{equation}
	    \nonumber
		\alert<2>{\sum_{j=1}^{N_s}}  \alert<2>{u_j}   \int_{\Omega_e}
		\nabla \alert<2>{\phi_j} \cdot \nabla \alert<2>{\phi_i} \;dx
		- \int_{\Omega_e}  f\alert<2>{\phi_i} \;dx
		\hspace{.15in},\hspace{.15in} i = 1,\ldots,N_s
	  \end{equation}
	}
      }
      \visible<3>
      {
	\item{
	  This can be expressed naturally in matrix notation as
	\begin{equation}
	  \nonumber
	  \bv{K^e} \bv{U^e} - \bv{F^e} 
	\end{equation}
	}
      }
  \end{itemize}
 \end{frame}



%% \frame%[t]
%%     {
%%   \frametitle{Poisson Equation}
%%   \begin{itemize}    
%%   \item
%%     {
%%       \visible<1->
%%       {
%% 	The element integrals \ldots
%%       }
%%       \visible<2->
%%       {
%% 	are written in terms of the local ``$\alert<2>{\phi_i}$'' basis functions \ldots
%%       }
%%       \visible<3>
%%       {
%% 	which can be expressed naturally in matrix notation.
%% 	%element ``stiffness matrix'' $\alert{\bv{K_e}}$
%% 	%and ``load vector'' $\alert{\bv{F_e}}$. 
%%       }
%%     }
%%   \end{itemize}
%%     \begin{eqnarray}
%%       %\begin{center}
%% 	\nonumber
%% 	%\begin{array}{c}
%% 	\int_{\Omega_e} \left[ \nabla u^h \cdot \nabla v^h - fv^h \right] dx
%% 	\hspace{.75in} \\ \nonumber
%% 	  \visible<2->
%% 	      {
%% 		\Downarrow \hspace{1.5in} \\ \nonumber
%% 		%
%% 		\alert<2>{\sum_{j=1}^{N_s}}  \alert<2>{u_j}   \int_{\Omega_e}
%% 		\nabla \alert<2>{\phi_j} \cdot \nabla \alert<2>{\phi_i} \;dx
%% 		- \int_{\Omega_e}  f\alert<2>{\phi_i} \;dx
%% 		\hspace{.15in},\hspace{.15in} i = 1,\ldots,N_s \\ \nonumber
%% 	      }
%% 	      \visible<3>
%% 	      {\Downarrow \hspace{1.5in} \\ \nonumber
%% 		%
%% 		\bv{K_e} \bv{U_e} - \bv{F_e} \hspace{1.25in}
%% 	      }
%% 	%\end{array}
%%       %\end{center}
%%     \end{eqnarray}
%%     }

\subsection*{Global Linear System}
\begin{frame}%[<+->]
  %  \frametitle{Poisson Equation}
  \begin{itemize}
    \visible<1->{
    \item{
      The entries of the element stiffness matrix are the integrals
      \begin{equation}
	\nonumber
	\bv{K}^e_{ij} := 
	\int_{\Omega_e}
	\nabla \phi_j \cdot \nabla \phi_i \;dx
      \end{equation}
    }
    }
    \visible<2->{
    \item{ While for the element right-hand side we have 
      \begin{equation}
	\nonumber
	\bv{F}^e_{i} := 
	\int_{\Omega_e} f \phi_i \;dx
      \end{equation}
    }
    }
    \visible<3>{
    \item{ The element stiffness matrices and right-hand sides can be ``assembled'' to 
      obtain the global system of equations
      \begin{equation}
	\nonumber
	\bv{K} \bv{U} = \bv{F}
      \end{equation}    
    }
    }
  \end{itemize}
\end{frame}

\subsection*{Reference Element Map}


\begin{frame}[t]
%  \frametitle{Poisson Equation}
  \begin{block}{}
    \begin{itemize}    
  \item{
    The integrals are performed on a ``reference'' element $\alert<1>{\hat{\Omega}_e}$
    }
  \end{itemize}
  \end{block}
  %\vspace{-.3in}
  %\begin{center}   %Note: \centering is what makes the tables ``wiggle'' during slide transitions
  %% Three separate tabular elements.  The first column is an empty, fixed-width column designed
  %% to center the table without using centering commands.
    \only<1>
    {
    \begin{tabular}{p{.125\textwidth}ccc} \\
      &
      \includegraphics[width=.2\textwidth]{physical_element}&
      \includegraphics[width=.2\textwidth]{map}&
      \includegraphics[width=.15\textwidth]{reference_element_red}
    \end{tabular}
    }
    %
    \only<2>
    {
    \begin{tabular}{p{.125\textwidth}ccc} \\ 
      &
      \includegraphics[width=.2\textwidth]{physical_element}&
      \includegraphics[width=.2\textwidth]{map_red}&
      \includegraphics[width=.15\textwidth]{reference_element}
    \end{tabular}
    }
    %
    \only<3>
    {
    \begin{tabular}{p{.125\textwidth}ccc} \\ 
      &
      \includegraphics[width=.2\textwidth]{physical_element}&
      \includegraphics[width=.2\textwidth]{map}&
      \includegraphics[width=.15\textwidth]{reference_element}
    \end{tabular}
    }
    
%%     %% All in one table
%%     \begin{tabular}{ccc} \\ 
%%     %\fbox{
%%       \includegraphics[width=.2\textwidth]{physical_element}
%%     %}
%%        &
%%   \only<1,3->
%%   {
%%        \includegraphics[width=.2\textwidth]{map}
%%   }
%%   \only<2>
%%   {
%%        \includegraphics[width=.2\textwidth]{map_red}
%%   }
%%        &
%%   \only<1>
%%   {
%%        \includegraphics[width=.15\textwidth]{reference_element_red}
%%        }
%%   \only<2->
%%   {
%%        \includegraphics[width=.15\textwidth]{reference_element}
%%        }
%%      \end{tabular}
  %\end{center}


  \only<2>
      {
	\begin{block}{}
	\begin{itemize}    
	\item{
	  The Jacobian of the map $\alert{x(\xi)}$ is $\alert{J}$.
	}
	\end{itemize}
	\end{block}
	\begin{equation}
	  \nonumber
	  \bv{F}^e_{i} = \int_{\Omega_e} f \phi_i dx
	  =  \int_{\alert{\hat{\Omega}_e}}
	  f (\alert{x(\xi)}) \phi_i \alert{|J|} d\alert{\xi}
	\end{equation}
      }

\only<3>
{
  \begin{block}{}
  \begin{itemize}    
  \item{
    %The gradients are transformed
    Chain rule: 
    $\nabla 
    = J^{-1}\nabla_{\!\xi}
    := \alert{\hat{\nabla}_{\!\xi}}$
  }
  \end{itemize}
  \end{block}
  \begin{equation}
    \nonumber
    \bv{K}^e_{ij} =
    \int_{\Omega_e}
    \nabla \phi_j \cdot \nabla \phi_i \;dx =
    \int_{\hat{\Omega}_e}
    \alert{\hat{\nabla}_{\!\xi}} \phi_j \cdot
    \alert{\hat{\nabla}_{\!\xi}} \phi_i \;|J| d\xi
  \end{equation}
}
\end{frame}

\subsection*{Element Quadrature}
    
\begin{frame}[t]
%	\frametitle{Poisson Equation}
	\begin{block}{}
	\begin{itemize}    
	\item{
	  The integrals on the ``reference'' element are approximated via numerical
	  quadrature.
	}
	  \visible<2->
	      {
	      \item{The quadrature rule has $\alert{N_q}$ points
		``$\alert{\xi_q}$'' and weights ``$\alert{w_q}$''.}
	      }
	\end{itemize}
	\end{block}
\only<3>
{
	\begin{eqnarray}
	  \nonumber
%	  \only<3-4>
%	      {
		\bv{F}^e_{i} &=&
		\int_{\hat{\Omega}_e} f \phi_i |J| d\xi
		\\ \nonumber
%	      }
%	      \only<4>
%		  {
		    &\approx&
		    \alert{\sum_{q=1}^{N_q}}
		    f(x(\alert{\xi_q})) \phi_i(\alert{\xi_q})
		    |J(\alert{\xi_q})| \alert{w_q}
%		  }
	\end{eqnarray}
}

\only<4>
{
	\begin{eqnarray}
	  \nonumber
%	  \only<5-6>
%	      {
		\bv{K}^e_{ij} &=&
		\int_{\hat{\Omega}_e}
		\hat{\nabla}_{\!\xi}\phi_j \cdot
		\hat{\nabla}_{\!\xi}\phi_i \;|J| d\xi
		\\ \nonumber
%	      }
%	      \only<6>
%		  {
		    &\approx&
		    \alert{\sum_{q=1}^{N_q}}
		    \hat{\nabla}_{\!\xi} \phi_j(\alert{\xi_q}) \cdot
		    \hat{\nabla}_{\!\xi} \phi_i(\alert{\xi_q})
		    |J(\alert{\xi_q})| \alert{w_q}
%		  }
	\end{eqnarray}
}
\end{frame}

\subsection*{\libMesh{} Quadrature Point Data}
\begin{frame}[t]
%	\frametitle{Poisson Equation}
	\begin{block}{}
	\begin{itemize}    
	\item{ \libMesh{} provides the following variables at
	  each quadrature point $q$
	}
%% 	\item{``\texttt{JxW[q]}'' = $|J(\xi_q)| w_q$
%% 	  %the scalar value of the element Jacobian map times
%% 	  %the quadrature rule weight
%% 	}
	\end{itemize}
	\end{block}
	
	\begin{center}
	  \renewcommand{\arraystretch}{1.3}
	\begin{tabular}{|l|l|l|} \hline
	  \textbf{Code} & \textbf{Math} & \textbf{Description} \\ \hline
	  \texttt{JxW[q]}
	  & $|J(\xi_q)| w_q$
	  & Jacobian times weight
	  \\ \hline
	  \texttt{phi[i][q]}
	  & $\phi_i(\xi_{q})$
	  & value of $i^{th}$ shape fn.\
	  \\ \hline
	  \texttt{dphi[i][q]}
	  & $\hat{\nabla}_{\!\xi} \phi_i (\xi_q)$
	  & value of $i^{th}$ shape fn.\ gradient
	  \\ \hline
	  \texttt{xyz[q]}
	  & $x(\xi_q)$
	  & location of $\xi_q$ in physical space
	  \\ \hline
	  \end{tabular}
	\end{center}
	  
%      } %end frame
\end{frame}

\subsection*{Matrix Assembly Loops}
\begin{frame}[fragile,t]  
%  \frametitle{Poisson Equation}
	\begin{block}{}
	  \begin{itemize}    
	  \item{ The \libmesh{} representation of the matrix and
	    rhs assembly is similar to the mathematical statements.
	  }
	  \end{itemize}
	\end{block}
\small
\begin{semiverbatim}
for (q=0; q<Nq; ++q) 
  for (i=0; i<Ns; ++i) \{
    \alert<2>{Fe(i)   += \alert<3>{JxW[q]}*\alert<4>{f(xyz[q])}*\alert<5>{phi[i][q]};}
    
    for (j=0; j<Ns; ++j)
      \alert<6>{Ke(i,j) += \alert<7>{JxW[q]}*(\alert<8>{dphi[j][q]*dphi[i][q]});}
  \}
\end{semiverbatim}
\only<2-5>
{
  \begin{equation}
    \nonumber
    \bv{F}^e_{i} = 
    \sum_{q=1}^{N_q}
    \alert<4>{f(x(\xi_q))}
    \alert<5>{\phi_i(\xi_q)}
    \alert<3>{|J(\xi_q)| w_q}
  \end{equation}
}
\only<6->
{
  \begin{equation}
  \nonumber
  \bv{K}^e_{ij} =
  \sum_{q=1}^{N_q}
  \alert<8>{
    \hat{\nabla}_{\!\xi} \phi_j(\xi_q) \cdot
    \hat{\nabla}_{\!\xi} \phi_i(\xi_q)
    }
  \alert<7>{|J(\xi_q)| w_q}
  \end{equation}
}
\end{frame}


\begin{frame}[allowframebreaks]
  \lstinputlisting[basicstyle=\tiny\ttfamily]{snippets/poisson_eqn.cxx}
\end{frame}
 

\frame
{
  \Large
  \begin{block}{}
    \center{\bf A Complete Program:}
    \center{\texttt{poisson\_ex1}}
  \end{block}
}



\begin{frame}[fragile]
  \frametitle{Poisson class definition}

  \begin{lstlisting}
// headers omitted for brevity
class Poisson : public System::Assembly
{
public:
  Poisson (EquationSystems &es_in) :
    es (es_in)
  {}

  void assemble ();

  Real exact_solution (const Real x,
                       const Real y,
                       const Real z = 0.) const
  {
    static const Real pi = acos(-1.);

    return cos(.5*pi*x)*sin(.5*pi*y)*cos(.5*pi*z);
  }

private:
  EquationSystems &es;
};
  \end{lstlisting}
\end{frame}


\begin{frame}[allowframebreaks]
  \frametitle{Poisson \texttt{main()}}
  \lstinputlisting[basicstyle=\tiny\ttfamily]{tutorial/poisson_ex1/main.C}
\end{frame}


\begin{frame}[allowframebreaks]
  \frametitle{Poisson \texttt{main()}}
  \lstinputlisting[basicstyle=\tiny\ttfamily]{tutorial/poisson_ex1/poisson_problem.C}
\end{frame}


\begin{frame}[fragile]
  \frametitle{Running the program}
    \begin{block}{Running the program}
    \begin{lstlisting}[language=bash]
# copy the example

$ make

# run the example in 2D with 20 elements in each direction
$ ./example-opt -d 2 -n 20 

# run the example in 2D with 20 elements in each direction
$ ./example-opt -d 3 -n 20 
    \end{lstlisting}
  \end{block}
\end{frame}

\frame
{
  \frametitle{Output}
  \begin{center}
    \includegraphics[height=0.8\textheight]{tutorial/poisson_ex1/screen}
  \end{center}
} 

\frame
{
  \Large
  \begin{block}{}
    \center{\bf Extension: Multithreaded Assembly:}
    \center{\texttt{poisson\_threaded}}
  \end{block}
}



\begin{frame}[fragile,shrink]
  \frametitle{Poisson class definition}

  \begin{lstlisting}
// headers omitted for brevity
class Poisson : public System::Assembly
{
public:
  Poisson (EquationSystems &es_in) :
    es (es_in)
  {}

  void assemble ();

  void operator()(const ConstElemRange &range) const;

  Real exact_solution (const Real x,
                       const Real y,
                       const Real z = 0.) const
  {
    static const Real pi = acos(-1.);

    return cos(.5*pi*x)*sin(.5*pi*y)*cos(.5*pi*z);
  }

private:
  EquationSystems &es;

  mutable Threads::spin_mutex assembly_mutex;
};
  \end{lstlisting}
\end{frame}



\begin{frame}[fragile,shrink]
  \frametitle{Threaded Poisson assembly}

  \begin{lstlisting}
#include "poisson_problem.h"

void Poisson::assemble ()
{
  const MeshBase& mesh = es.get_mesh();

  ConstElemRange assembly_elem_range (mesh.active_local_elements_begin(),
                                      mesh.active_local_elements_end());

  Threads::parallel_for (// the range over which we will perform threaded operations
                         assembly_elem_range,

                         // the function object to apply to each element in the range
                         *this);
}

void Poisson::operator()(const ConstElemRange &range) const
{
  ...

  // insert the local (per-thread) element matrix/vector into
  // the global matrix/vector.  This is a shared object, so we
  // must be careful to lock for exclusive access.
  {
    Threads::spin_mutex::scoped_lock lock(assembly_mutex);
    
    system.matrix->add_matrix (Ke, dof_indices);
    system.rhs->add_vector    (Fe, dof_indices);
  }
}
  \end{lstlisting}
\end{frame}
\begin{frame}[fragile]
  \frametitle{Running the program}
    \begin{block}{Running the program}
    \begin{lstlisting}[language=bash]
# copy the example

$ make

# run the example in 2D with 20 elements in each direction
$ ./example-opt -d 2 -n 20 

# run the example in 2D with 20 elements in each direction
$ ./example-opt -d 3 -n 20 --n_threads=1
$ ./example-opt -d 3 -n 20 --n_threads=2
$ ./example-opt -d 3 -n 20 --n_threads=4

    \end{lstlisting}
  \end{block}
\end{frame}

      

\begin{frame}[fragile]  % {
  \frametitle{Convection-Diffusion Equation}
	\begin{block}{}
	  \begin{itemize}    
	  \item{The matrix assembly routine for the linear convection-diffusion equation,
	    \begin{equation}
	      \nonumber
	      -\alert<2>{k}\Delta u + \alert<3>{\bv{b} \cdot \nabla u} = f
	    \end{equation}
	  }
	  \end{itemize}
	\end{block}
\small
\begin{semiverbatim}
for (q=0; q<Nq; ++q) 
  for (i=0; i<Ns; ++i) \{
    Fe(i)   += JxW[q]*f(xyz[q])*phi[i][q];
    
    for (j=0; j<Ns; ++j)
      Ke(i,j) += JxW[q]*(\alert<2>{k}*(dphi[j][q]*dphi[i][q]) 
                      + (\alert<3>{b*dphi[j][q]})*phi[i][q]);
  \}
\end{semiverbatim}
\end{frame}

\subsection{Coupled Variables}

\begin{frame}[t]  % {
  \frametitle{Stokes Flow}
  \begin{block}{}
    \begin{itemize}    
    \item{For multi-variable systems like Stokes flow,
      \begin{equation}
	\begin{array}{rcl}
	  \nonumber
	  %\frac{\partial \bv{u}}{\partial t} +
	  %\left(\bv{u} \cdot \nabla\right) \bv{u} +
	  \nabla p - \nu \Delta \bv{u}  &=& \bv{f}
	  \\
	  \nonumber
	  \nabla \cdot \bv{u} &=& 0
	\end{array}  \hspace{.25in}  \in \hspace{.1in} \Omega \subset \mathbb{R}^2
      \end{equation}
    }
\vspace{-.25in}
      
    \item{The element stiffness matrix concept can extended to include sub-matrices
      \begin{eqnarray}
	\nonumber
	\label{eqn:Ke_stokes}
	\left[
	  \begin{array}{cc|c}
	    \alert<2>{K^e_{u_1 u_1}}   & K^e_{u_1 u_2}             &  K^e_{u_1 p}        \\
	    K^e_{u_2 u_1}              & \alert<3>{K^e_{u_2 u_2}}  &  K^e_{u_2 p} \\ \hline
	    K^e_{p u_1}                & \alert<4>{K^e_{p u_2}}    &  K^e_{p p}      \\
	  \end{array}
	  \right]
	\left[
  \begin{array}{c}
    U^e_{u_1} \\
    U^e_{u_2}\\ \hline
    U^e_{p}
  \end{array}
  \right]-
\left[
  \begin{array}{c}
    \alert<6>{F^e_{u_{1}}} \\
    \alert<7>{F^e_{u_{2}}} \\ \hline
    F^e_{p}
  \end{array}
  \right]
      \end{eqnarray}
    }


      \item
	{
	  \only<1-4>	      {We have an array of submatrices:}
	      \only<1>	      {\texttt{Ke[ ][ ]}}
	      \only<2>	      {\hspace{-0.05in}\texttt{Ke[\alert<2>{1}][\alert<2>{1}]}}
	      \only<3>	      {\hspace{-0.1in}\texttt{Ke[\alert<3>{2}][\alert<3>{2}]}}
	      \only<4>        {\hspace{-0.15in}\texttt{Ke[\alert<4>{3}][\alert<4>{2}]}}

      \only<5->          { 	  And an array of right-hand sides: }
 	\only<5> {\texttt{Fe[]}.}
	\only<6> {\hspace{-0.05in}\texttt{Fe[\alert{1}]}.}
	\only<7> {\hspace{-0.1in}\texttt{Fe[\alert{2}]}.}
	}

	
%%       \only<1>
%% 	  {
%% 	  \item{
%% 	    We have an array of submatrices
%% 	    \texttt{Ke[ ][ ]}.
%% 	  }
%% 	  }

%% 	  \only<2>
%% 	  {
%% 	  \item{
%% 	    We have an array of submatrices
%% 	    \texttt{Ke[\alert<2>{1}][\alert<2>{1}]}.
%% 	  }
%% 	  } 
%%           \only<3>
%%           {
%%  	  \item{
%%  	    In this case, we have an array of submatrices
%%  	    \texttt{Ke[\alert<3>{2}][\alert<3>{2}]}.
%%  	  }
%% 	  }
%%           \only<4>
%%           {
%%  	  \item{
%%  	    In this case, we have an array of submatrices
%%  	    \texttt{Ke[\alert<4>{3}][\alert<4>{2}]}.
%%  	  }
%% 	  }
%%           \only<5>
%%           {
%%  	  \item{
%%  	    And an array of right-hand sides
%%  	    \texttt{Fe[]}.
%%  	  }
%% 	  }
%% 	  \only<6>
%% 	  {
%%  	  \item{
%%  	    And an array of right-hand sides
%%  	    \texttt{Fe[\alert{1}]}.
%%  	  }
%% 	  }
    \end{itemize}
  \end{block}
\end{frame}




\begin{frame}[fragile]  % {
  \frametitle{Stokes Flow}
  \begin{block}{}
    \begin{itemize}    
    \item{The matrix assembly can proceed in essentially the same way.}
    \item{For the momentum equations:}
    \end{itemize}
  \end{block}
  \small
\begin{semiverbatim}
for (q=0; q<Nq; ++q) 
  \alert{for (d=0; d<2; ++d)}
    for (i=0; i<Ns; ++i) \{
      Fe\alert{[d]}(i) += JxW[q]*f(xyz[q],\alert{d})*phi[i][q];
      
      for (j=0; j<Ns; ++j)
        Ke\alert{[d][d]}(i,j) +=
	            JxW[q]*\alert{nu}*(dphi[j][q]*dphi[i][q]);
    \}
\end{semiverbatim}
\end{frame}

\section{Essential BCs}
% Auto-generate the TOC slide(s)
\begin{frame}
  \tableofcontents[currentsection]
  %\tableofcontents
\end{frame}



\subsection*{Essential Boundary Data}
\begin{frame}[t]
  %\vspace{-.2in}
  \begin{block}{
      %Essential Boundary Data
    }
  \begin{itemize}
  \item {Dirichlet boundary conditions can be enforced after 
    the global stiffness matrix $\bv{K}$ has been assembled}
  \item This usually involves
    \begin{enumerate}
    \item<1-> placing a ``1'' on the main diagonal of the
      global stiffness matrix
    \item<2-> zeroing out the row entries
    \item<3-> placing the Dirichlet
      value in the rhs vector
    \item<4-> subtracting off the column entries from the rhs
    \end{enumerate}
  \end{itemize}
  \end{block}
  \visible<5->{
    \vspace{-0.1in}
  \begin{equation}
    \nonumber
      \begin{bmatrix}
	k_{11} & k_{12} & k_{13} & .  \\
	k_{21} & k_{22} & k_{23} & .  \\
	k_{31} & k_{32} & k_{33} & .  \\
	  .    &   .    &    .   & .  
      \end{bmatrix},
      \begin{bmatrix}
	f_{1}  \\
	f_{2}  \\
	f_{3}  \\
	  .     
      \end{bmatrix} \rightarrow
      \begin{bmatrix}
	1      & 0      & 0      & 0  \\
	0      & k_{22} & k_{23} & .  \\
	0      & k_{32} & k_{33} & .  \\
	  0    &   .    &    .   & .  
      \end{bmatrix},
      \begin{bmatrix}
	g_{1}  \\
	f_{2} - k_{21}g_1  \\
	f_{3} - k_{31}g_1  \\
	  .     
      \end{bmatrix}      
  \end{equation}}

\end{frame}



\begin{frame}[c]
%\begin{block}{}
  \begin{itemize}[<+->]
    \item {Cons of this approach :
      \begin{itemize}[<+->]
      \item {Works for an interpolary finite element basis
	but not in general.}
	
      \item {May be inefficient to change individual entries once the global matrix is assembled.}
      \end{itemize}
      }
    \item {Need to enforce boundary conditions for
      a generic finite element basis \emph{at the element stiffness matrix level}.}

    %\item Solution: ``Penalty'' Boundary Conditions
  \end{itemize}
%  \end{block}
\end{frame}


\subsection*{Penalty Formulation}
\begin{frame}[c]
%\begin{block}{}
  \begin{itemize}[<+->]
  \item {One solution is the ``penalty'' boundary formulation}
    %
  \item {A term is added to the standard weighted residual statement
    \begin{equation}
      \nonumber
      (F( u ), v)
      + \underbrace{\frac{1}{\epsilon} \int_{\partial \Omega_D} (u-u_D)v \; dx}_{\text{penalty term}} =
      0 \hspace{.3in} \forall v \in \mathcal{V}
    \end{equation}
  }
    %
  \item {Here $\epsilon \ll 1$ is chosen so that, in floating point arithmetic,
    $\frac{1}{\epsilon} + 1 = \frac{1}{\epsilon}$.}
    %
  \item {This weakly enforces $u=u_D$ on the Dirichlet boundary, and works for
    general finite element bases.}

%%   \item It requires a few additional calculations (edge/face integrals) but is more
%%     efficient than modifying row entries after assembly
  \end{itemize}
%  \end{block}
\end{frame}



\begin{frame}[fragile]
  \begin{block}{}
  \texttt{\libMesh{}} provides:
  \begin{itemize}
  \item {A quadrature rule with \texttt{Nqf} points and \texttt{JxW\_f[]}}
  \item {A finite element coincident with the boundary face that has % \texttt{Nf}
    shape function values \texttt{phi\_f[][]}}
  \end{itemize}
  \end{block}
\small
  \begin{semiverbatim}
for (qf=0; qf<Nqf; ++qf) \{
  for (i=0; i<Nf; ++i) \{
    Fe(i) += JxW_f[qf]*
      \alert<2>{penalty}*\alert<3>{uD(xyz[q])}*phi_f[i][qf];
	
    for (j=0; j<Nf; ++j)
      Ke(i,j) += JxW_f[qf]*
        \alert<2>{penalty}*phi_f[j][qf]*phi_f[i][qf];
  \}
\}
    \end{semiverbatim}

\end{frame}

\section{Some Extensions}
% Auto-generate the TOC slide(s)
\begin{frame}
  \tableofcontents[currentsection]
  %\tableofcontents
\end{frame}



\subsection*{Time-Dependent Problems}
\begin{frame}%[t]
  \only<1>{
  }
%%   \begin{block}{}
%%     The weighted residual statement provides the connection between the mathematical
%%     statement of the problem and the computer code implementation of the problem:
%%   \end{block}

  %\begin{block}{}
  \begin{itemize}
    \only<1>
	{
	\item{For linear problems, we have already seen how
	  the weighted residual statement
	  leads directly to a sparse linear system of equations
	  \begin{equation}
	    \nonumber
	    \bv{K} \bv{U} = \bv{F}
	  \end{equation}
	  %which can be solved via Krylov subspace iterative methods.
	}
	}
    \only<2>
	{
	\item{For time-dependent problems, 
	  \begin{equation}
	    \nonumber
	    \frac{\partial u}{\partial t} = F(u)
	  \end{equation}
	}
	\item{we also need a way to advance the
	  solution in time, e.g. a $\theta$-method
	  \begin{eqnarray}
	    \nonumber
	    \left( \frac{ u^{n+1} - u^n}{\Delta t}, v^h\right) &=& \left(F(u_{\theta}), v^h\right)
	    \hspace{.1in} \forall v^h \in \mathcal{V}^h
	    %+ \mathcal{O}(\Delta t^{p(\theta)})
	    \\ \nonumber
	    u_{\theta} &:=& \theta u^{n+1} + (1-\theta)u^n
	  \end{eqnarray}
	\item{Leads to $\bv{K} \bv{U} = \bv{F}$ at \emph{each timestep}.}
	}
	}
  \end{itemize}
%\end{block}
\end{frame}





\subsection*{Nonlinear Problems}
\begin{frame}
  \begin{itemize}
	\item{For nonlinear problems, typically a sequence of linear problems must be solved, e.g.
	  for Newton's method
	  \begin{equation}
	    \nonumber
	    (F'( u^k ) \delta u^{k+1}, v) = -(F( u^k ), v) 
	  \end{equation}
	  where $F'( u^k )$ is the linearized (Jacobian) operator associated with
	  the PDE.	}

	\item{Must solve $\pdv{\bv{F}}{\bv{U}}\delta\bv{U} = -\bv{F}$ (Inexact Newton method) at \emph{each iteration step}.}
  \end{itemize}
\end{frame}




\section*{Reference}
\begin{frame}[t]
  \begin{block}{}
    \begin{itemize}
    \item{
      %Some applications are shown from:
      %\\
      %\vspace{.5in}
      %\begin{block}{}
      B. Kirk, J. Peterson, R. Stogner and G. Carey, ``libMesh: a C++
      library for parallel adaptive mesh refinement/coarsening
      simulations'',  \emph{Engineering with Computers}, vol.~22, no.~3--4, p.~237--254, 2006.
      %\end{block}
      }
    \item{
      Public site, mailing lists, SVN tree, examples, etc.:
\texttt{http://libmesh.sf.net/}
      }
    \end{itemize}
  \end{block}
\end{frame}



\end{document}






% LocalWords:  rcl fv Nonlinearity
